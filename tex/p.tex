\section{P}


\entry{pleasing}{pleas-ing}{adj.}{To popularize something means to make a lot of people interested in it and able to enjoy it. } --- \textit{We also experiment with single-image super-resolution, where replacing a per-pixel loss with a perceptual loss gives visually \textbf{pleasing} results.}

\entry{popularize}{popular-ize}{vt.}{Something that is pleasing gives you pleasure and satisfaction. } --- \textit{The problem of visual domain adaptation was introduced along with a pairwise metric transform solution by Saenko et al. (2010) and was further \textbf{popularized} by the broad study of visual dataset bias (Torralba \& Efros, 2011)}


\entry{presumably}{presum-ab-ly}{adv.}{ If you say that something is presumably the case, you mean that you think it is very likely to be the case, although you are not certain.} --- \textit{Future improvements to GANs can \textbf{presumably} be expected to yield/generate further improvements to semi-supervised learning.}


\entry{prevail}{prevail}{vi.}{If a proposal, principle, or opinion prevails, it gains influence or is accepted, often after a struggle or argument.} --- \textit{This pipeline has  \textbf{prevailed} on detection benchmarks since the Selective Search work through the current leading results on PASCAL VOC, COCO, and ILSVRC detection all based on Faster R-CNN...}


\entry{prevailing}{prevail-ing}{adj.}{The prevailing wind in an area is the type of wind that blows over that area most of the time.} --- \textit{Yang et.al. provide an exhaustive evaluation of  \textbf{prevailing} techniques prior to the widespread adoption of convolutional networks.}

\entry{prominent}{prominent}{adj.}{Something that is prominent is very noticeable or is an important part of something else.} --- \textit{A \textbf{prominent} class of probabilistic models of images are restricted Boltzmann machines and their deep variants.}


\entry{promise}{promise}{n-UNcount}{If someone or something shows promise, they seem likely to be very good or successful.} --- \textit{Nonetheless, they demonstrate the \textbf{promise} of our approach as a generic commodity tool for image-to-image translation problems.}

\entry{promising}{promis-ing}{adj.}{Someone or something that is promising seems likely to be very good or successful.} --- \textit{The results in this paper suggest that conditional adversarial networks are a \textbf{promising} approach for many image-to-image translation tasks, especially those involving highly structured graphical outputs.}

