\section{R}

\entry{remarkable}{re-mark-able}{adj.}{Someone or something that is remarkable is unusual or special in a way that makes people notice them and be surprised or impressed.} --- \textit{Given the fact that bottom-up methods have always performed less accurately than the top-down methods, our results are \textbf{remarkable}.}



\entry{render}{render}{vt.}{You can use render with an adjective that describes a particular state to say that someone or something is changed into that state. For example, if someone or something makes a thing harmless, you can say that they render it harmless.} --- \textit{We find that applying orthogonal regularization to the generator \textbf{renders} it amenable.}


\entry{resemble}{resemble}{vt.}{If one thing or person resembles another, they are similar to each other.} --- \textit{In all cases, the generated images look sharp and \textbf{resemble} natural images.}

 
