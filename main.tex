\documentclass{article}

% if you need to pass options to natbib, use, e.g.:
% \PassOptionsToPackage{numbers, compress}{natbib}
% before loading nips_2018

% ready for submission
% \usepackage{nips_2018}

% to compile a preprint version, e.g., for submission to arXiv, add
% add the [preprint] option:
\usepackage[preprint]{nips_2018}

% to compile a camera-ready version, add the [final] option, e.g.:
% \usepackage[final]{nips_2018}

% to avoid loading the natbib package, add option nonatbib:
% \usepackage[nonatbib]{nips_2018}

\usepackage[utf8]{inputenc} % allow utf-8 input
\usepackage[T1]{fontenc}    % use 8-bit T1 fonts
\usepackage{hyperref}       % hyperlinks
\usepackage{url}            % simple URL typesetting
\usepackage{booktabs}       % professional-quality tables
\usepackage{amsfonts}       % blackboard math symbols
\usepackage{nicefrac}       % compact symbols for 1/2, etc.
\usepackage{microtype}      % microtypography
\usepackage{multicol}
% \usepackage{xeCJK}%据 @李清  改
% \setCJKmainfont[BoldFont=SimHei,ItalicFont=KaiTi]{SimSu}
% \setCJKmonofont{KaiTi}
% \usepackage{CJK}
% \usepackage{apacite}
% \usepackage[authoryear]{natbib}



% \usepackage[backend=biber,style=authoryear]{biblatex}
% \addbibresource{example.bib}

\usepackage{fancyhdr} % Required for modifying headers and footers
\fancyhead[L]{\textsf{\rightmark}} % Top left header
\fancyhead[R]{\textsf{\leftmark}} % Top right header
\renewcommand{\headrulewidth}{1.4pt} % Rule under the header
\fancyfoot[C]{\textbf{\textsf{\thepage}}} % Bottom center footer
\renewcommand{\footrulewidth}{1.4pt} % Rule under the footer
\pagestyle{fancy} % Use the custom headers and footers throughout the document

\newcommand{\entry}[4]{\markboth{#1}{#1}\textbf{#1}\ {(#2)}\ \textit{#3}\ $\bullet$\ {#4}} 

\title{Dictionary of Scientific Paper Writing in Computer Vision for Non-native Speakers }

% The \author macro works with any number of authors. There are two
% commands used to separate the names and addresses of multiple
% authors: \And and \AND.
%
% Using \And between authors leaves it to LaTeX to determine where to
% break the lines. Using \AND forces a line break at that point. So,
% if LaTeX puts 3 of 4 authors names on the first line, and the last
% on the second line, try using \AND instead of \And before the third
% author name.

\author{
  Fei Wang\thanks{Work started at September 5th 2018, when FW was a visiting student at Carnegie Mellon University.}\\
%     \\
 School of Computer Science, Xi'an Jiaotong University\\
%  , Xi'an 710049, China\\
 The Robotics Institute, Carnegie Mellon University
 \\
%   Pittsburgh 15213, 15213 \\
  \texttt{feiw2.ri@gmail.com} \\
  %% examples of more authors
%   \And
%   Zeyi Huang\thanks{work} \\
%   Carnegie Mellon University \\
% %    \\
%   \texttt{zeyih@andrew.cmu.edu} \\
  %% \AND
  %% Coauthor \\
  %% Affiliation \\
  %% Address \\
  %% \texttt{email} \\
  %% \And
  %% Coauthor \\
  %% Affiliation \\
  %% Address \\
  %% \texttt{email} \\
  %% \And
  %% Coauthor \\
  %% Affiliation \\
  %% Address \\
  %% \texttt{email} \\
}

\begin{document}
% \nipsfinalcopy is no longer used

\maketitle

% \begin{abstract}
%   The abstract paragraph should be indented \nicefrac{1}{2}~inch
%   (3~picas) on both the left- and right-hand margins. Use 10~point
%   type, with a vertical spacing (leading) of 11~points.  The word
%   \textbf{Abstract} must be centered, bold, and in point size 12. Two
%   line spaces precede the abstract. The abstract must be limited to
%   one paragraph.
% \end{abstract}
%  \tableofcontents

%-----------------------------------------------------------------
%	SECTION A
%-----------------------------------------------------------------
\section{A}

% \begin{multicols}{2}

\entry{absence}{ab-sen-ce}{N-SING.}{The absence of something from a place is the fact that it is not there or does not exist.} ---~\textit{We present an approach for learning to translate an image from a source domain X to a target domain Y \textbf{in the absence of} paired examples.}


\entry{advance}{advance}{v.}{If you advance a cause, interest, or claim, you support it and help to make it successful.} ---~\textit{In particular, our model is capable of synthesizing 2K resolution videos of street scenes up to 30 seconds long, which significantly \textbf{advances} the state-of-the-art of video synthesis.}
 
 \entry{amenable}{amen-able}{adj.}{If you are amenable to something, you are willing to do it or accept it.} ---~\textit{We
find that applying orthogonal regularization to the generator renders it \textbf{amenable}...}

\entry{appealing}{appeal-ing}{adj.}{Someone or something that is appealing is pleasing and attractive.} ---~\textit{We observe that translations on training data are often more \textbf{appealing} than those on test data.}


\entry{approximately}{approximate-ly}{adv.} ---~\textit{One can \textbf{approximately} model all conditionals by training a family of conditional models that share parameters.}

\entry{aware}{aware}{adj.}{If you are aware of something, you know about it.} ---~\textit{We are not \textbf{aware} of any work making use of similarity metrics for machine learning, except a recent pre-print of Ridgeway et al. (2015).}








%-----------------------------------------------------------------
%	SECTION B
%-----------------------------------------------------------------
\section{B}

\entry{branch}{branch}{v.}{Divide into two or more branches so as to form a fork.} --- \textit{Multi person pose estimation solutions \textbf{branched} out as bottom-up and top-down methods.} 


\section{C}


\entry{chaotic}{chao-tic}{adj.}{Something that is chaotic is in a state of complete disorder and confusion.} --- \textit{This may be because minor structural errors are more visible in maps, which have rigid geometry, than in aerial photographs, which are more \textbf{chaotic}.}

\entry{compelling}{compel-l-ing}{adj.}{If you describe something such as a film or book, or someone's appearance, as compelling, you mean you want to keep looking at it or reading it because you find it so interesting.} --- \textit{The cGANs can produce \textbf{compelling} colorizations (first two rows), but have a common failure mode of producing a grayscale result (last row).}

\entry{complementary}{comple-ment-ary}{adj.}{Complementary things are different from each other but make a good combination.} --- \textit{ Further, we show that adaptation at both the pixel and representation level can offer \textbf{complementary} improvements with joint pixel-space and feature adaptation leading to the highest performing model for digit classification tasks.}


\entry{considerably}{consider-able-ly}{adv.}{Relatively or Pretty.} --- \textit{Here we show some of the most successful results in our test set - average performance is \textbf{considerably} worse.}

\entry{contemporary}{con-tempor-ary}{adj.}{Contemporary things are modern and relate to the present time.} --- \textit{Our experiments confirm that domain adaptation can benefit greatly from cycle-consistent pixel transformations, and that this is especially important for pixel-level semantic segmentation with \textbf{contemporary} FCN architectures.}


\entry{continuation}{continua-tion}{n.}{Something that is a continuation of something else is closely connected with it or forms part of it.} --- \textit{Like many deep generative models, GANs have previously been applied to semi-supervised learning [13, 14], and our work can be seen as a \textbf{continuation} and refinement of this effort.}

\entry{couple}{couple}{v.}{If you say that one thing produces a particular effect when it is coupled with another, you mean that the two things combine to produce that effect.} --- \textit{Through carefully-designed generator and discriminator architectures, \textbf{coupled} with spatio-temporal adversarial objective, we achieve high-resolution, photorealistic, temporally coherent video results on a diverse set of input formats including segmentation masks, sketches and poses.}

\entry{cumbersome}{cumbersome}{adj.}{A cumbersome system or process is very complicated and inefficient.} ---~\textit{Using a learned video synthesis model, one can generate realistic videos without explicitly specifying scene geometry, materials, lighting, and their dynamics, which would be \textbf{cumbersome} but necessary when using standard graphics rendering techniques.}



\section{D}

\entry{devise}{devise}{vt.}{If you devise a plan, system, or machine, you have the idea for it and design it.} --- \textit{Training could be accelerated greatly by \textbf{devising} better methods for coordinating G and D.}

\entry{differentiate}{different-i-ate}{vt.}{If you differentiate between things or if you differentiate one thing from another, you recognize or show the difference between them.} --- \textit{The generator aims to produce realistic synthetic data so that the discriminator cannot \textbf{differentiate} between real and synthetic data.}


\entry{delineate}{delineate}{vt.}{If you delineate something such as an idea or situation, you describe it or define it, often in a lot of detail.} --- \textit{We compare our objective with that of Taskonomy [42] to \textbf{delineate} the difference.}


\entry{deterministic}{determin-istic}{adj.}{Deterministic ideas or explanations are based on determinism.} --- \textit{Recently, stochastic neural networks have become popular, and \textbf{deterministic} networks are being used for image generation tasks.}



\section{E}

\entry{eliminate}{eliminate}{vt.}{To eliminate something, especially something you do not want or need, means to remove it completely.} ---\textit{The fundamental improvement in
speed comes from \textbf{eliminating}  bounding box proposals and the subsequent pixel or feature resampling stage.}


\entry{elusive}{elusive}{adj.}{Something or someone that is elusive is difficult to find, describe, remember, or achieve.} ---\textit{Despite recent progress in generative image modeling, successfully generating high-resolution, diverse samples from complex datasets such as ImageNet remains
an \textbf{elusive} goal.}

\entry{encourage}{encourage}{vt.}{If something encourages a particular activity or state, it causes it to happen or increase.} --- \textit{We introduce the following techniques that are heuristically motivated to \textbf{encourage} convergence.}

\entry{engineer}{engine-er}{verb.}{When a vehicle, bridge, or building is engineered, it is planned and constructed using scientific methods.} ---\textit{We note that their method was specifically \textbf{engineered} to do well on colorization}

\entry{excel}{excel}{vi.}{If someone excels in something or excels at it, they are very good at doing it.} ---\textit{Deep neural networks \textbf{excel} at learning from large amounts of data, but can be poor at generalizing learned knowledge to new datasets or environments.}


\entry{explicit}{explicit}{adj.}{Something that is explicit is expressed or shown clearly and openly, without any attempt to hide anything.} ---\textit{
That is faster than
the previous state-of-the-art for single shot detectors (YOLO), and significantly
more accurate, in fact as accurate as slower techniques that perform \textbf{explicit} region
proposals and pooling (including Faster R-CNN).}


\section{F}


\entry{frame}{frame}{verb.}{When a picture or photograph is framed, it is put in a frame.} --- \textit{Many classic problems can be \textbf{framed} as image transformation tasks, where a system receives some input image and transforms it into an output image.}



\section{G}


\entry{Babble}{bab-uh l}{Verb}{Talk rapidly and continuously in a foolish, excited, or incomprehensible way.}



\section{H}

\entry{high-end}{high-end}{adj.}{ High-end products, especially electronic products, are the most expensive of their kind.} --- \textit{While accurate, these approaches have been too computationally intensive for embedded system and, even with \textbf{high-end} hardware, too slow for real-time applications.}

\section{I}

\entry{imminent}{imminent}{adj.}{If you say that something is imminent, especially something unpleasant, you mean it is almost certain to happen very soon.} --- \textit{When collapse to a single mode is \textbf{imminent}, the gradient of the discriminator may point in similar directions for many similar points.}


\entry{intractable}{in-tract-able}{adj.}{If you say that a person, problem, or device is tractable, you mean that they can be easily controlled or dealt with.} --- \textit{Such models generally have \textbf{intractable} likelihood functions and therefore require numerous approximations to the likelihood gradient.}


\entry{intriguing}{intrigu-ing}{adj.}{If you describe something as intriguing, you mean that it is interesting or strange.} --- \textit{Szegedy et al first discovered an \textbf{intriguing} weakness of deep neural networks in the context of image classification.}


\section{J}

\begin{multicols}{2}

% \entry{Babble}{bab-uh l}{Verb}{Talk rapidly and continuously in a foolish, excited, or incomprehensible way.}


\end{multicols}

\section{K}

\begin{multicols}{2}

% \entry{Babble}{bab-uh l}{Verb}{Talk rapidly and continuously in a foolish, excited, or incomprehensible way.}


\end{multicols}

\section{L}


\entry{leading}{lead-ing}{adj.}{The leading group, vehicle, or person in a race or procession is the one that is at the front.} --- \textit{We first provide a quantitative comparison against \textbf{leading} methods in Sec. 4.1.}



\section{M}


\entry{mitigate}{mitigate}{vt.}{To mitigate something means to make it less unpleasant, serious, or painful.} --- \textit{We propose a class of similarity metrics, that \textbf{mitigate} this problem.}



\section{N}

\begin{multicols}{2}

% \entry{Babble}{bab-uh l}{Verb}{Talk rapidly and continuously in a foolish, excited, or incomprehensible way.}


\end{multicols}

\section{O}

\begin{multicols}{2}

% \entry{Babble}{bab-uh l}{Verb}{Talk rapidly and continuously in a foolish, excited, or incomprehensible way.}


\end{multicols}

\section{P}


\entry{pleasing}{pleas-ing}{adj.}{To popularize something means to make a lot of people interested in it and able to enjoy it. } --- \textit{We also experiment with single-image super-resolution, where replacing a per-pixel loss with a perceptual loss gives visually \textbf{pleasing} results.}

\entry{popularize}{popular-ize}{vt.}{Something that is pleasing gives you pleasure and satisfaction. } --- \textit{The problem of visual domain adaptation was introduced along with a pairwise metric transform solution by Saenko et al. (2010) and was further \textbf{popularized} by the broad study of visual dataset bias (Torralba \& Efros, 2011)}


\entry{presumably}{presum-ab-ly}{adv.}{ If you say that something is presumably the case, you mean that you think it is very likely to be the case, although you are not certain.} --- \textit{Future improvements to GANs can \textbf{presumably} be expected to yield/generate further improvements to semi-supervised learning.}


\entry{prevail}{prevail}{vi.}{If a proposal, principle, or opinion prevails, it gains influence or is accepted, often after a struggle or argument.} --- \textit{This pipeline has  \textbf{prevailed} on detection benchmarks since the Selective Search work through the current leading results on PASCAL VOC, COCO, and ILSVRC detection all based on Faster R-CNN...}


\entry{prevailing}{prevail-ing}{adj.}{The prevailing wind in an area is the type of wind that blows over that area most of the time.} --- \textit{Yang et.al. provide an exhaustive evaluation of  \textbf{prevailing} techniques prior to the widespread adoption of convolutional networks.}

\entry{prominent}{prominent}{adj.}{Something that is prominent is very noticeable or is an important part of something else.} --- \textit{A \textbf{prominent} class of probabilistic models of images are restricted Boltzmann machines and their deep variants.}


\entry{promise}{promise}{n-UNcount}{If someone or something shows promise, they seem likely to be very good or successful.} --- \textit{Nonetheless, they demonstrate the \textbf{promise} of our approach as a generic commodity tool for image-to-image translation problems.}

\entry{promising}{promis-ing}{adj.}{Someone or something that is promising seems likely to be very good or successful.} --- \textit{The results in this paper suggest that conditional adversarial networks are a \textbf{promising} approach for many image-to-image translation tasks, especially those involving highly structured graphical outputs.}


\section{Q}

\begin{multicols}{2}

% \entry{Babble}{bab-uh l}{Verb}{Talk rapidly and continuously in a foolish, excited, or incomprehensible way.}


\end{multicols}

\section{R}

\entry{remarkable}{re-mark-able}{adj.}{Someone or something that is remarkable is unusual or special in a way that makes people notice them and be surprised or impressed.} --- \textit{Given the fact that bottom-up methods have always performed less accurately than the top-down methods, our results are \textbf{remarkable}.}



\entry{render}{render}{vt.}{You can use render with an adjective that describes a particular state to say that someone or something is changed into that state. For example, if someone or something makes a thing harmless, you can say that they render it harmless.} --- \textit{We find that applying orthogonal regularization to the generator \textbf{renders} it amenable.}


\entry{resemble}{resemble}{vt.}{If one thing or person resembles another, they are similar to each other.} --- \textit{In all cases, the generated images look sharp and \textbf{resemble} natural images.}

 

\section{S}

\entry{sidestep}{side-step}{vt.}{If you sidestep a problem, you avoid discussing it or dealing with it.} --- \textit{We propose a new generative model estimation procedure that \textbf{sidesteps} these difficulties.}

\entry{spurious}{spur-ious}{adj.}{Something that is spurious seems to be genuine, but is false.} 
--- \textit{Even a slight departure from a network's training domain can cause it to make \textbf{spurious} predictions and significantly hurt its performance.}

\entry{striking}{strik-ing}{adj.}{Something that is striking is very noticeable or unusual.} --- \textit{So far, the most \textbf{striking} successes in deep learning have involved discriminative models, usually those that map a high-dimensional, rich sensory input to a class label.}

\entry{subtle}{subtle}{adj.}{Something that is subtle is not immediately obvious or noticeable.} 
--- \textit{In many cases, these modifications can be so \textbf{subtle} that a human observer does not even notice the modification at all, yet the classifier still makes a mistake.}

\entry{superior}{superior}{adj.}{If you describe something as superior, you mean that it is good, and better than other things of the same kind.} 
--- \textit{When both U-Net and encoder-decoder are trained with an L1 loss, the U-Net again achieves the \textbf{superior} results.}



\entry{surge}{surge}{n.}{A surge is a sudden large increase in something that has previously been steady, or has only increased or developed slowly.} 
--- \textit{Recently there has been a \textbf{surge} of interest in training neural networks to generate images.}


\entry{susceptible}{suscepti-ble}{adj.}{If you are susceptible to something or someone, you are very likely to be influenced by them.} 
--- \textit{They showed that despite their high accuracies, modern deep
networks are surprisingly \textbf{susceptible} to adversarial attacks
in the form of small perturbations to images that remain
(almost) imperceptible to human vision system.}



\section{T}

\entry{tackle}{tackle}{verb.}{If you tackle a difficult problem or task, you deal with it in a very determined or efficient way.} --- \textit{Although these approaches \textbf{tackle} the multi-modal image synthesis problem, they are unsuitable for our image manipulation task mainly for two reasons.}

\section{U}

\entry{unfortunate}{un-fortun-ate}{adj.}{If you describe something that has happened as unfortunate, you think that it is inappropriate, embarrassing, awkward, or undesirable.} ~~~ \textit{Batch normalization is very helpful, but for GANs has a few \textbf{unfortunate} side effects.} 
\section{V}

\entry{vulnerable}{vulner-able}{adj.}{Something that is vulnerable can be easily harmed or affected by something bad.} ~~~ \textit{Most existing machine learning classifiers are highly  \textbf{vulnerable} to adversarial examples.}


\section{W}

\begin{multicols}{2}

% \entry{Babble}{bab-uh l}{Verb}{Talk rapidly and continuously in a foolish, excited, or incomprehensible way.}


\end{multicols}

\section{X}

\begin{multicols}{2}

% \entry{Babble}{bab-uh l}{Verb}{Talk rapidly and continuously in a foolish, excited, or incomprehensible way.}


\end{multicols}

\section{Y}

\begin{multicols}{2}

% \entry{Babble}{bab-uh l}{Verb}{Talk rapidly and continuously in a foolish, excited, or incomprehensible way.}


\end{multicols}

\section{Z}

\begin{multicols}{2}

% \entry{Babble}{bab-uh l}{Verb}{Talk rapidly and continuously in a foolish, excited, or incomprehensible way.}


\end{multicols}


\end{document}
